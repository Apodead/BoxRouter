Welcome to Box\+Router2.\+0, a new global router for ultimate routability.

This is 64 bit version.

\href{https://www.cerc.utexas.edu/utda/download/BoxRouter.htm}{\texttt{ Box\+Router2.\+0 Website}}



 \section*{Tutorial}

\subsection*{1. What do you have in this folder and What do you need?}

\subsubsection*{You have\+:}

\paragraph*{a) Box\+Router2.\+0 Source Code\+:}


\begin{DoxyItemize}
\item B\+Box.\+h Boundary.\+h Box\+Router.\+h Design.\+h Grid.\+h G\+Router.\+h I\+L\+P\+Solver.\+h Maze\+Heap.\+h Net.\+h Object.\+h Param.\+h Pin.\+h Point.\+h Segment.\+h Util.\+h Wire.\+h
\item B\+Box.\+cpp Boundary.\+cpp Box\+Router.\+cpp Design.\+cpp Grid.\+cpp G\+Router.\+cpp I\+L\+P\+Solver.\+cpp Maze\+Heap.\+cpp Net.\+cpp Object.\+cpp Param.\+cpp Pin.\+cpp Point.\+cpp Segment.\+cpp Util.\+cpp Wire.\+cpp dmp2res.\+cpp res2dmp.\+cpp
\item makefile
\item fix\+\_\+mps.\+pl
\end{DoxyItemize}

\subsubsection*{You need\+:}

\paragraph*{b) Software packages (free for academic research) used in Box\+Router2.\+0}


\begin{DoxyItemize}
\item I\+: F\+L\+U\+TE\+:
\item II\+: glpk I\+LP solver\+:
\item I\+II\+: S\+T\+Lport\+:

\href{https://sourceforge.net/projects/stlport/files/STLport/}{\texttt{ Histrory Release}}
\item IV\+: M\+O\+S\+EK\+:

\href{https://www.mosek.com/downloads/list/4/}{\texttt{ History Release}}
\end{DoxyItemize}

\paragraph*{c) Complied executable file}

\begin{DoxyVerb}res2dmp.x dmp2res.x br_ispd07.x
\end{DoxyVerb}


\subsubsection*{To help you test, we also put one example here\+:}

\begin{DoxyVerb}ibm01.par, the BoxRouter2.0 configuration file.
\end{DoxyVerb}



\begin{DoxyItemize}
\item bench/ibm01.\+modified.\+txt -\/$>$ input file, one of I\+S\+P\+D98 benchmarks.
\item bench/ibm01.\+modified.\+stt -\/$>$ the Steiner tree generated by F\+L\+U\+TE.
\item bench/ibm01.\+modified.\+txt.\+log-\/$>$ log file.
\item bench/ibm01.\+modified.\+res -\/$>$ output result.
\item bench/ibm01.\+modified.\+dmp -\/$>$ dump file.
\item bench/ibm01.\+modified.\+res.\+map -\/$>$ the final congestion map.
\end{DoxyItemize}

\subsection*{2. How to compile the code ?}

\subsubsection*{Please first make sure you setup those software packages correctly.}


\begin{DoxyItemize}
\item a) Download F\+L\+U\+TE package and set it up in your Box\+Router2.\+0 folder.

For your convenience, Flute files which we are using currently can be also downloaded from \href{http://www.cerc.utexas.edu/utda/download/FLUTE.tar.gz}{\texttt{ http\+://www.\+cerc.\+utexas.\+edu/utda/download/\+F\+L\+U\+T\+E.\+tar.\+gz}}. You can download it and extract into your Box\+Router2.\+0 folder.
\item b) Download glpk package and set it up in your Box\+Router2.\+0 folder.

For your convenience, glpk-\/4.\+10 which we are using currently can be also downloaded from \href{http://www.cerc.utexas.edu/utda/download/glpk-4.10.tar.gz}{\texttt{ http\+://www.\+cerc.\+utexas.\+edu/utda/download/glpk-\/4.\+10.\+tar.\+gz}}. You can download it and extract into your Box\+Router2.\+0 folder.
\item c) Download S\+T\+L\+P\+O\+RT package and set it up in your Box\+Router2.\+0 folder.
\item d) Download M\+O\+S\+EK package and set it up in your Box\+Router2.\+0 folder.
\end{DoxyItemize}

\subsubsection*{Then, please compile the code as\+:}


\begin{DoxyItemize}
\item a) make

Generate the Box\+Router2.\+0 executable file br\+\_\+ispd07.\+x
\item b) make dmp2res\+:

Generate the executable file dmp2res.\+x
\item c) make res2dmp\+:

Generate the executable file res2dmp.\+x
\end{DoxyItemize}

\subsection*{3. How to run the Box\+Router2.\+0?}

\begin{DoxyVerb}./br_ispd07.x configuration_file 
\end{DoxyVerb}


Here we provide one example ibm01.\+par for configuration\+\_\+file.

\subsection*{4. What are the meanings of those parameters in our configuration file?}


\begin{DoxyItemize}
\item a) I\+N\+P\+U\+T\+\_\+\+F\+I\+LE input\+\_\+filename

Specify the input circuit in I\+S\+PD 2007 Routing Contest format.
\item b) T\+R\+E\+E\+\_\+\+F\+I\+LE tree\+\_\+filename

Input Steiner tree. If tree\+\_\+filename is not found,Box\+Router2.\+0 will call F\+L\+U\+TE to make Steiner trees for you. The format of the Steiner tree file will be discussed later.
\item c) O\+U\+T\+P\+U\+T\+\_\+\+F\+I\+LE output\+\_\+filename

Specify the routing result output file and the format can be found in our Box\+Router2.\+0 website.
\item d) D\+U\+M\+P\+\_\+\+F\+I\+LE dump\+\_\+filename

The dump file is used to store intermediate routing solution in binary format.

If the file dump\+\_\+filename exists before you run the Box\+Router2.\+0, the Box\+Router2.\+0 will just read the dump file to get intermediate solution and start from there , skipping certain steps.

If the file dump\+\_\+filename doesn\textquotesingle{}t exist, the Box\+Router2.\+0 will do the whole routing from the first beginning and write several intermediate results into dump files.
\item e) M\+U\+L\+T\+I\+\_\+\+L\+A\+Y\+ER a

a = 0 disables multi layer option. a = 1 enables multi layer option.
\item f) P\+R\+E\+R\+O\+U\+T\+I\+NG a

a = 0 will N\+OT perform the prerouting stage. a = 1 will perform the prerouting stage.
\item g) B\+O\+X\+R\+O\+U\+T\+I\+NG a

a = 0 will N\+OT perform the boxrouting stage. a = 1 will perform the boxrouting stage.
\item h) B\+O\+X\+R\+O\+U\+T\+I\+N\+G\+\_\+\+S\+T\+EP n

n is the number of variables which can be affordably solved by G\+L\+PK.
\item i) B\+O\+X\+R\+O\+U\+T\+I\+N\+G\+\_\+\+I\+LP n

n is used to specify the algorithm which is used in boxrouting stage.

n = 0;use mazerouting. n = 1\+:use I\+LP solver and set I\+LP objective is max. n = 2\+:use I\+LP solver and set I\+LP objective is min. n = 3\+:use pattern routing. n = 4\+:use hybrid I\+LP mode. n = 5\+:set I\+LP objective max and use round up to achieve the solution.
\item j) I\+L\+P\+\_\+\+S\+O\+L\+V\+ER n

n = 0\+:G\+L\+PK is used. n = 1;M\+O\+S\+EK is used.
\item k) R\+E\+R\+O\+U\+T\+I\+NG a

a = 0 will N\+OT perform the rerouting stage. a = 1 will perform the rerouting stage.
\item l) R\+E\+R\+O\+U\+T\+I\+N\+G\+\_\+\+C\+O\+U\+NT n

n is the number of most rerouting functions.
\item m) R\+E\+R\+O\+U\+T\+I\+N\+G\+\_\+\+R\+E\+P\+E\+AT n

n is the number of rerouting for reducing wirelength.
\item n) R\+E\+R\+O\+U\+T\+I\+N\+G\+\_\+\+S\+T\+EP n

n is used to control the number of wires to be ripup on congestion edges.
\item o) R\+E\+R\+O\+U\+T\+I\+N\+G\+\_\+\+P\+U\+SH p
\item p) R\+E\+R\+O\+U\+T\+I\+N\+G\+\_\+\+S\+T\+U\+CK s

The above two parameters are used together to control the routing cost function. e.\+g. Cost = a$\ast$(b.$^\wedge$(P\+U\+S\+T\+\_\+\+V\+A\+L\+UE))+1; a\&b are constants. p is the starting P\+U\+S\+H\+\_\+\+V\+A\+L\+UE; s is predefined S\+T\+U\+C\+K\+\_\+\+V\+A\+L\+UE;

During the rerouting stage, for every new P\+U\+S\+H\+\_\+\+V\+A\+L\+UE, we set S\+T\+U\+C\+K\+\_\+\+V\+A\+L\+UE 0. If we use this P\+U\+SH V\+A\+L\+UE and during following rerouting, we can not improve our result for continuous S\+T\+U\+C\+K\+\_\+\+V\+L\+A\+UE times, we will increase p by 1.
\item q) M\+A\+Z\+E\+R\+O\+U\+T\+I\+N\+G\+\_\+\+M\+A\+R\+G\+IN n

n is used to control mazerouting to avoid too much searching space.
\item r) R\+E\+L\+A\+Y\+E\+R\+I\+NG a

a = 0 will N\+OT perform Box\+Router2.\+0 layer assignment algorithm. a = 1 will perform Box\+Router2.\+0 layer assignment algorithm.
\end{DoxyItemize}

You can always comment out some line in the configuration file by putting \# in the beginning of the line as \begin{DoxyVerb}#MULTI_LAYER 
\end{DoxyVerb}


The Box\+Router2.\+0 will assign default value accordingly.

\subsection*{5. How does the T\+R\+EE F\+I\+LE format looks like?}

You can specify your own tree topology and write it into file as following format. The Box\+Router2.\+0 perform the routing algorithm based on you file.


\begin{DoxyCode}{0}
\DoxyCodeLine{net[name from the input] [net ID from the input] [\# of wires]}
\DoxyCodeLine{([x11],[y11])-(x[12],y[12])}
\DoxyCodeLine{([x21],[y21])-(x[22],y[22])}
\DoxyCodeLine{....}
\DoxyCodeLine{....}
\DoxyCodeLine{!}
\end{DoxyCode}


for example


\begin{DoxyCode}{0}
\DoxyCodeLine{net10122 3 6 // a net with name"net10122" has netID 3 and the Steiner tree has 6 wires}
\DoxyCodeLine{17 61 - 18 61 // the 1st wire is from (17,61 to 18.61)}
\DoxyCodeLine{23 61 - 23 62}
\DoxyCodeLine{25 62 - 23 62}
\DoxyCodeLine{18 61 - 21 61}
\DoxyCodeLine{21 61 - 21 62}
\DoxyCodeLine{23 62 - 21 62}
\DoxyCodeLine{!}
\end{DoxyCode}
 

 \section*{Authors}


\begin{DoxyItemize}
\item Minsik Cho, Kun Yuan, Katrina Lu and David Z. Pan
\end{DoxyItemize}



 \section*{L\+I\+C\+E\+N\+SE}

Please see L\+I\+C\+E\+N\+SE for information.



 \section*{Awards \& Publications}


\begin{DoxyItemize}
\item 2nd Place in 3D Routing Contest
\item 3rd Place in 2D Routing Contest
\item Minsik Cho and David Z. Pan , \char`\"{}\+Box\+Router\+: A New Global Router Based on Box Expansion and Progressive I\+L\+P\char`\"{}, Proc. Design Automation Conference (D\+AC), July, 2006. (Nominated for Best Paper Award, 12 out of 865 submissions) \subsection*{$\ast$ Minsik Cho, Katrina Lu, Kun Yuan, David Z. Pan, \char`\"{}\+Box\+Router 2.\+0\+: Architecture and Implementation of a Hybrid and Robust Global Router\char`\"{}, Proc. I\+E\+E\+E/\+A\+CM Int\textquotesingle{}l Conference on Computer-\/\+Aided Design (I\+C\+C\+AD), November, 2007. }
\end{DoxyItemize}

\section*{Disclaimer}

Box\+Router2.\+0 was originally released in 2007. Many interfaces, for example M\+O\+S\+EK, may have been changed. Please use the software only for reference. 